\documentclass{article}
\textheight = 20cm
\textwidth = 18cm
\topmargin = -2cm
\oddsidemargin= -1cm
\parindent = 0mm
\usepackage{amsmath,amssymb,amsfonts,latexsym}
\usepackage{graphicx}

\begin{document}
\section{Instalacion de QT CREATOR}
{\bf Instalacion.} Para poder instalar QT CREATOR nos ayudamos con un tutorial colgado en la web, los pasos de la instalacion sumamente sencillos a continuacion se los detalla.
\begin{enumerate}
\item{\bf} Descargamos el Qt Creator de la siguiente pagina: http://qt.nokia.com/downloads.
\item{\bf} Una vez que el programa se haya descargado completamente, ingresaremos a la carpeta donde se encuentra dicho programa y hacemos doble clic para ejecutarlo.
\item{\bf} Al ejecutarse la ventana de instalacion nos presenta una portada de bienvenida, hacemos clic en next.
\item{\bf} Se despliega otra presentacion en la cual nos pide que aceptemos la licencia de uso del programa, seleccionamos: I accept the terms of the License Agreement y luego damos clic en next.
\item{\bf} Nos muestra los enlaces de descarga del Microsoft Console Debugger (CDB) el que se usa en el IDE de Visual Studio y el GDB que es parte de MinGW, luego damos clic en next.
\item{\bf} Se despliega otra presentacion en la cual nos pide que seleccionemos lo componentes que queremos que instale el programa, lo dejamos asi por default, luego damos clic en next.
\item{\bf} Luego nos muestra la ruta en donde queremos que se instale el programa, asi lo dejamos, luego damos clic en next.
\item{\bf} Luego nos muestra el proceso de instalacion del programa, dependiedo de nuestra computadora este proceso va a tomar solo unos minutos o mas.
\item{\bf} Y finalmente presenta la ultima ventana de instalacion del programa, si dejamos activada la casilla Run Qt Creator, al darle clic a Finish la instalacion abra terminado y automaticamente Qt Creator se ejecutara.

\end{enumerate}
\end{document}